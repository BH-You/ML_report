\documentclass{llncs}

\usepackage{amsmath,amssymb}
\usepackage{alg}
\usepackage{booktabs}
\PassOptionsToPackage{hyphens}{url}\usepackage{hyperref}
\usepackage{graphicx}
\usepackage{subfigure}
\usepackage{xspace}
\usepackage{color}

\Urlmuskip=0mu plus 1mu

\renewcommand\dblfloatpagefraction{.99}
\renewcommand\dbltopfraction{.9}
\renewcommand\floatpagefraction{.99}
\renewcommand\topfraction{.9}

\newcommand{\name}{\texttt{G-NPM}\xspace}
\newcommand{\npm}{\texttt{NPM}\xspace}
\newcommand{\todo}[1]{\texttt{\color{red}TODO:#1}}
\begin{document}


\title{G-NPM: 
Local Citation Recommendation System 
with Global Features}

\author{Byeonghyeon You, Seongmin Lee, Junbeom Lee, Jinhan Kim}

\institute{Korea Advanced Institute of Science and Technology\\Republic of Korea}

\maketitle

\begin{abstract}
\todo{JB}

\end{abstract}

\section{Introduction}
\label{sec:introduction}
As academic communities have published about the millions of research papers, finding appropriate citations become burdensome for researchers. Citation recommendation systems[] have cut down the overload of finding citations by automatically suggesting papers which might be related to the topic.


There are two main categories in citation recommendation, global and local citation recommendataion.
The first category, global citation recommendation, recommends a list of candidate papers by examining some part of the entire paper. In addition to exmaine the text, global citation recommendation systems use the external features such as information of author, venue, citation count, and h-index. These external features can make recommendation model more accurate.
The second category, local citation recommendation, recommends a list of candidate papers by just looking a sentence which is called citation context. Local citation recommendation systems only take citation context as an input, suggest papers which is relevant to that context. Intuitively, it is more practical to use than global citation recommendation systems.


Although the current state-of-the-art local citation recommendation system[] outperformed other approaches[], it showed relatively poor in recommending well-cited papers. Since global and local has own advantages, in this project, we propose the new citation recommendation system which is local citation recommendation system with global external features. We use citation count as a global external feature.


\section{Background}
This section describes the two citation recommendation models.

\subsection{Neural Probabilistic Model}
Neural Probabilistic Model, \npm ~\cite{Huang:2015:NPM:2886521.2886655}

NPM(Neural Probabilistic Model) is local citation recommendation model that learns the citing paper with given citation contexts. Training is separated into two parts, word representation learning and document representation learning. To learn words, they used negative sampling[] that is used to learn the distribution of words. To learn documents with words, they used noise-contrastive estimation[] that is used to learn the distribution of words and documents.

\subsection{Literature Search Model}
Literature Search Model~\cite{Bethard:2010:ICL:1871437.1871517}

\section{\name: A new citation recommendation system}
\todo{SM}


\section{Experiments}
\subsection{Research Questions}
\todo{SM}

\begin{enumerate}
\item Does G-NPM use global information(citation count) to improve the efficiency of citation recommendation system?
\item Is the G-NPM a model that can be applied competently on systems with small input data?
\end{enumerate}
\subsection{Setup}
\todo{JB}

\subsection{Evaluation Metric}
\todo{JB}

\section{Results}
\todo{BH}

\section{Discussion}
\todo{BH}

\section{Conclusion}
\label{sec:Conclusion}
\todo{BH}


\bibliographystyle{splncs03}
\bibliography{ref}


\end{document}
